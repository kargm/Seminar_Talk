\documentclass{beamer}
\usepackage{german}

\usepackage{flashmovie}
\usepackage{multimedia}
%\usepackage{movie15}
\graphicspath{{../../../pictures/src/}}
% commands for the translation of the document are shown after \end{document}

%% mode specific adjustments

\usetheme{TUMias}
\setbeamertemplate{headline}[tumias]
%\usecolortheme{squid}
%\setbeamertemplate{headline}[ias]%[english]

% f�r Garching oder Freising gibt es noch weitere M�glichkeit f�r die Kopfzeile
%\useoutertheme{shadedhead}

\mode<beamer>
{
  \AtBeginSection[]
  {
    \begin{frame}<beamer>
      \frametitle{Outline}
      \tableofcontents[currentsection,currentsubsection]
    \end{frame}
  }
}

%% General Information
\title[Expectations in HRI] % (optional, use only with long paper titles)
{Towards Expectation-Based Failure Recognition for Human Robot Interaction}

\subtitle{TUM-IAS and LAAS-RIA Spring Workshop} % (optional)

\author % (optional, use only with lots of authors)
{Michael Karg}
% - Use the \inst{?} command only if the authors have different
%   affiliation.

\institute[TU München] % (optional, but mostly needed)
{
  
  Department of Computer Science\\
  Intelligent Autonomous Systems Group
}
% - Use the \inst command only if there are several affiliations.
% - Keep it simple, no one is interested in your street address.

\date[TUM-IAS and LAAS-RIA Spring Workshop] % (optional)
{11 July 2011}


\subject{Talks}
% This is only inserted into the PDF information catalog. Can be left
% out.

\newcommand{\texcmd}[1]{{\ttfamily\textbackslash #1}}

%\newcommand{\todo}[1]{\textbf{\textsc{\textcolor{red}{(TODO: #1)}}}}


\begin{document}
\definecolor{blue}{rgb}{0,0.396, 0.741}
\maketitle

\begin{frame}
  \frametitle{Outline}
  \tableofcontents
  % You might wish to add the option [pausesections]
\end{frame}

\section[Adapto Project]{The Adapto Project}

\begin{frame}{The Adapto Project}
  \small
  \begin{itemize}
    %\item Part of Intelligent Autonomous Systems group of Prof. Beetz at TUM
    \item Intelligent Autonomous Systems Group of Prof. Michael Beetz
    \item Adapto project under the supervision of Dr. Alexandra Kirsch
    \item Goal: Make technical systems more comprehensive for users to support people in everyday activities
    \item Develop algorithms inspired by human behavior for dynamic, uncertain and human centered environments
    \item Use of planning, knowledge representation and machine learning
    %TODO: GRUPPENBILD
    %\item Failure recognition and avoidance using expectation models
    %\item Continual adaptation by learning prediction models
  \end{itemize}
  \begin{tabular}[h]{p{0.3\textwidth} p{0.3\textwidth} p{0.3\textwidth}}
     \begin{figure}[h]
    \includegraphics[scale=0.090]{para/hri_morse.png} 
  \end{figure}&
    \begin{figure}[h]
    \includegraphics[scale=0.062]{para/para_group.jpg} 
  \end{figure} &
   \begin{figure}[h]
    \includegraphics[scale=0.08]{para/jointtableset.jpg} 
  \end{figure}\\
  \end{tabular}
 
\end{frame}

\section[Expectations]{Expectations for Failure Recognition}

%\subsection{Why use Expectations?}
\begin{frame}{Why Use Expectations?}
  \begin{itemize}
  \item Not always clear: \textcolor{blue}{What is a failure?}
  \item Impossible for the programmer to anticipate all possible states and failures in the real world
  \item Expectations can enable us to \textcolor{blue}{detect unexpected events and failures} that have not been foreseen by the programmer
  \item Diagnosis using expectations can help to prevent errors and react to them thus making plans more \textcolor{blue}{flexible, reliable and general}
  \end{itemize}
\end{frame}

%\subsection{What are Expectations?}
\begin{frame}{What Are Expectations?}
\begin{itemize}
  \item Combination of different models that implicitly include predictions of robots environment, the human partner(s), the robot itself etc.
  \item Allow for a general understanding of ``normality'' 
\end{itemize}
 \begin{tabular}[h]{p{0.45\textwidth} p{0.45\textwidth}}
    \begin{figure}[h]
    \includegraphics[scale=0.095]{para/morse_scene_abnormal.png}
  \end{figure} &
   \begin{figure}[h]
    \includegraphics[scale=0.095]{para/morse_scene_normal.png} 
  \end{figure}\\
  \end{tabular}
\end{frame}

%\subsection{Expectations for Failure Recognition and Recovery}
\begin{frame}{Using Expectations for Failure Recognition and Recovery}
 \begin{figure}[h]
    \includegraphics[scale=0.95]{para/adapto_expectations_structure_2.png} 
  \end{figure}
%TODO: Plan adaptation in Bild!
\end{frame}

%\subsection{Challenges}
\begin{frame}{Challenges}
We see three main challenges:
 \begin{description}
\item [Representation] of expectations
\item [Modeling efficiency]: Generation of expectations
\item [Execution efficiency]: Validation of expectations
\end{description} 
 \begin{figure}[h]
%     \begin{tabular}{p{0.5\textwidth} p{0.4\textwidth}}
%      &
        \includegraphics[scale=0.1]{para/Jido_confused.png} \\
%     \end{tabular}
  \end{figure}
\end{frame}

%\subsection*{Challenges: Representation of Expectations}
\label{sec:representation}

\begin{frame}{Challenges: Representation of Expectations}
  \begin{itemize}
  \item Four categories: robot's plans, environment, human and robot itself
  \item Categories have dependencies among each other
  \item Expectations are time variant and have to be adaptable to changes
  \item Expectations should be capable of dealing with uncertainties
  \end{itemize}
  \begin{block}{Ideas:}
    Hierarchical, probabilistic hybrid automaton, high-level particle filter, probabilistic logic, fuzzy logic
  \end{block}
\end{frame}

%\subsection*{Challenges: Modeling Efficiency}
\label{sec:model}

\begin{frame}{Challenges: Modeling Efficiency}
  \begin{itemize}
  \item Generation of different models by learning from experiences
  \item Combination of models to generate expectations
  \item General/Abstract model, so not every possible failure state has to be modeled
  \item Learning from observations and experiences
  \end{itemize}
  \begin{block}{Ideas:}
  Ideas: Use abstract ontological and geometrical knowledge from available data bases combined with continual learning of experiences
  \end{block}
\end{frame}

%\subsection*{Challenges: Execution Efficiency}
\label{sec:execution}

\begin{frame}{Challenges: Execution Efficiency}
  \begin{itemize}
  \item Prerequisite: Real time
  \item Validation of expectations should allow to recognize ``abnormal'' behavior and trigger adequate reactions
  \item Continual learning and update of existing expectations while plan execution
  \end{itemize}
  \begin{block}{Ideas:}
    Validation of expectations depending on ``normality'' of the robots current world to limit the search space (Inspired by humans).
  \end{block}
\end{frame}

\section{Implementation and Evaluation}

%\subsection{CRAM Plan Language}
\label{sec:cram}

\begin{frame}{The CRAM Plan Language}
  \begin{itemize}
  \item Cognitive Robot Abstract Machine
  \item Plan-based control of autonomous robots
  \item CRAM plan language based on CommonLISP
  \item Complex failure handling
  \item Task synchronization, parallel execution, resource management
  \end{itemize}
  \begin{block}{CRAM-PL ROS-package:}
      \centering
      \url{http://www.ros.org/wiki/cram_pl}
  \end{block}
\end{frame} 


%\subsection{MORSE Simulator}
\begin{frame}{The MORSE Simulator}
\begin{itemize}
\item Modular Robot Simulation Engine
\item Modularity allows to regulate level of detail according to our needs (e.g. task planning)
\item Human model already available
\item Game mode for Human Robot Interaction scenarios
\item Compatible with ROS
\item Model of PR2 on the way
\end{itemize}
 \begin{figure}[h]
    \includegraphics[scale=0.076]{para/human_navigation_wide.png} 
  \end{figure}

\end{frame}

%\subsection*{An Intuitive Human Control Interface for MORSE}
\begin{frame}{An Intuitive Human Control Interface for MORSE}
 \begin{center}
   \movie[externalviewer]{\includegraphics[width=0.85\textwidth]{para/morse_human_interface.png}}{morse_human_interface.mp4}
 \end{center}
%    \begin{tabular}{p{0.9\textwidth}}	
%      \flashmoviex[width=9.5cm,height=7cm]{morse_human_interface.mp4}\\
%    \end{tabular}
%  \includemovie[poster,autoplay, controls=false,endat=time:12]{6cm}{4.5cm}{morse_human_interface.avi}
\end{frame}

\section{Summary}

\begin{frame}
  \frametitle<presentation>{Summary}

  \begin{itemize}
  \item Expectations in cooperative human robot interaction plans
  \item Abstract model of expectations about the robots plans, the environment, the human(s) and the robot itself
  \item Three main challenges: Representation, modeling efficiency, execution efficiency
  \item Implementation using CRAM Plan Language
  \item Evaluation using MORSE simulator and an intuitive human control interface
  \end{itemize}

\end{frame}

\begin{frame}{The end}
 \begin{center}
  Any questions?
\end{center}
\end{frame}

\end{document}


Making Slides:
pdflatex beamerexample.tex

Making Handout:
beamer-handout --frame true beamerexample.tex

